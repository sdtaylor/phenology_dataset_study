
\documentclass[a4paper,12pt]{article}

\usepackage{graphicx}
\usepackage{a4wide}
\usepackage{lineno}
    \linenumbers
%\usepackage{setspace}
%  \doublespacing
\usepackage[document]{ragged2e}

\begin{document}
\begin{center}
{\huge Paper title goes here \par} \newline

\textit{Shawn D. Taylor, Joan M. Meiners, Kristina Riemer, Michael C. Orr, Ethan P. White} \newline

{\Large \textbf{Supplementary materials}} \newline
Methods describing processing of the 4 LTER datasets
\end{center}

The four LTER datasets each had different protocols for recording phenology observations. Below are details for converting the data from each to a status based yes/no as used in the National Phenology Network. As in the NPN datasets, the julian day of year (DOY) used in modeling was the midpoint between each "yes" observation and the most recent "no" observation. 

\textbf{Harvard Forest} \newline
Observations here were recorded as relative percentage of flowering or budburst for individual plants. "Yes" observations budburst and flower where set to the DOY when the percentage of each tree had was greater than or equal to 10\%

\textbf{H.J. Andrews Experimental Forest} \newline
"yes" observations for budburst was the first DOY when an individual was marked as "bud break". "yes" observations for flowering was was when an indiviudal was first marked as "Flowers open". Note that each species has slightly different ordinal codes to make each of these events. 

\textbf{Jornada Experimental Range} \newline
Observations for this dataset represent, for each zone, the percent of plants for a particular species which is observed within each phenophase. "Yes" observations for flowers was the first DOY where the flower phenophase was 10\% or greater.

\textbf{Hubbard Brook} \newline
Observations for this dataset represent for each species the average, among 3 individuals, of an ordinal description of phenophase. \newline

0: winter conditions \newline
1: bud swelling \newline
2: small leaves or flowers \newline
3: leafs 1/2 of final length, leafs obscure half the sky as seen thru crown \newline
3.5: leaves 3/4 expanded, sky mostly obscured, crown not yet in summer condition \newline
4: fully expanded, canopy in summer conditions \newline

"Yes" observations for budburst were the DOY when the average value was greater than or equal to 1.6. This is the value where the 3 individuals most likely have ordinal observations of [1,2,2].

\end{document}}
